% Compile with Luatex

\documentclass{resume} % Use the custom resume.cls style
%\documentclass{twocolumn} % Use the custom resume.cls style
\usepackage{fontawesome,graphicx}
\usepackage{hyperref}
\usepackage{todonotes}
\usepackage{amsmath}
\usepackage{wasysym}
\usepackage{soul}
\usepackage[outline]{contour}% http://ctan.org/pkg/contour

% Google logo from https://tex.stackexchange.com/questions/292638/writing-google-logo-in-latex/292652
\usepackage{tikz}
\usetikzlibrary{svg.path}
\definecolor{gG}{RGB}{ 60, 186,  84}
\definecolor{gY}{RGB}{244, 194,  13}
\definecolor{gB}{RGB}{ 72, 133, 237}
\definecolor{gR}{RGB}{219,  50,  54}

\newlength\htG
\protected\def\google{\settoheight{\htG}{G}%
	\begin{tikzpicture}[yscale=-1,scale=(\htG/240pt),baseline=(baseline)]
	\fill[fill=gG] svg {m797.49 249.7h35.975v-240.75h-35.975z};
	\coordinate (baseline) at (current bounding box.south);
	\fill[fill=gB] svg {m246.11 116.18h-116.57v34.591h82.673c-4.0842 48.506-44.44 69.192-82.533 69.192-48.736 0-91.264-38.346-91.264-92.092 0-52.357 40.54-92.679 91.371-92.679 39.217 0 62.326 25 62.326 25l24.22-25.081s-31.087-34.608-87.784-34.608c-72.197-0.001-128.05 60.933-128.05 126.75 0 64.493 52.539 127.38 129.89 127.38 68.031 0 117.83-46.604 117.83-115.52 0-14.539-2.1109-22.942-2.1109-22.942z};
	\fill[fill=gR] svg {m341.6 91.129c-47.832 0-82.111 37.395-82.111 81.008 0 44.258 33.249 82.348 82.673 82.348 44.742 0 81.397-34.197 81.397-81.397 0-54.098-42.638-81.959-81.959-81.959zm0.47563 32.083c23.522 0 45.812 19.017 45.812 49.66 0 29.993-22.195 49.552-45.92 49.552-26.068 0-46.633-20.878-46.633-49.79 0-28.292 20.31-49.422 46.741-49.422z};
	\fill[fill=gY] svg {m520.18 91.129c-47.832 0-82.111 37.395-82.111 81.008 0 44.258 33.249 82.348 82.673 82.348 44.742 0 81.397-34.197 81.397-81.397 0-54.098-42.638-81.959-81.959-81.959zm0.47562 32.083c23.522 0 45.812 19.017 45.812 49.66 0 29.993-22.195 49.552-45.92 49.552-26.068 0-46.633-20.878-46.633-49.79 0-28.292 20.31-49.422 46.741-49.422z};
	\fill[fill=gB] svg {m695.34 91.215c-43.904 0-78.414 38.453-78.414 81.613 0 49.163 40.009 81.765 77.657 81.765 23.279 0 35.657-9.2405 44.796-19.847v16.106c0 28.18-17.11 45.055-42.936 45.055-24.949 0-37.463-18.551-41.812-29.078l-31.391 13.123c11.136 23.547 33.554 48.103 73.463 48.103 43.652 0 76.922-27.495 76.922-85.159v-146.77h-34.245v13.836c-10.53-11.347-24.93-18.745-44.04-18.745zm3.178 32.018c21.525 0 43.628 18.38 43.628 49.768 0 31.904-22.056 49.487-44.104 49.487-23.406 0-45.185-19.005-45.185-49.184 0-31.358 22.619-50.071 45.66-50.071z};
	\fill[fill=gR] svg {m925.89 91.02c-41.414 0-76.187 32.95-76.187 81.57 0 51.447 38.759 81.959 80.165 81.959 34.558 0 55.768-18.906 68.426-35.845l-28.235-18.787c-7.3268 11.371-19.576 22.484-40.018 22.484-22.962 0-33.52-12.574-40.061-24.754l109.52-45.444-5.6859-13.318c-10.58-26.08-35.26-47.86-67.92-47.86zm1.4268 31.413c14.923 0 25.663 7.9342 30.224 17.447l-73.139 30.57c-3.1532-23.667 19.269-48.017 42.915-48.017z};
	\end{tikzpicture}%
}

\usepackage[left=0.5in,top=1.0in,right=0.5in,bottom=1.0in]{geometry} % Document margins
\hypersetup{
	colorlinks=true,
	urlcolor=[rgb]{0.3, 0.3, 0.3},
}


\newcommand*{\img}[1]{%
	\raisebox{-.02\baselineskip}{%
		\includegraphics[
		height=\baselineskip,
		width=\baselineskip,
		keepaspectratio,
		]{#1}%
	}%
}
\newcommand*{\emoji}[1]{\img{./emoji/\imgpref#1.png}}
\newcommand*{\imgd}[1]{\img{./img/\imgpref#1.png}}

%\setul{0.3ex}{0.1ex}
\definecolor{ulcolor}{RGB}{230, 230, 230}
%\setulcolor{ulcolor}

%\newcommand*{\mybold}[1]{{\contourlength{0.1ex}\contournumber{10}\contour{ulcolor}{#1}}}
\definecolor{pinkunderline}{RGB}{100, 0, 100}
\newcommand*{\mybold}[1]{{\color{pinkunderline} #1}}
\newcommand*{\myboldq}[1]{{#1}}
%\newcommand*{\mybold}[1]{\ul{#1}}

%too unnoticeable
%\newcommand*{\mybold}[1]{{\em #1}}

\newcommand*{\logo}[1]{%
	\raisebox{-.02\baselineskip}{%
		\includegraphics[
		height=10pt,
		keepaspectratio,
		interpolate
		]{./img/\imgpref#1}%
	}%
}

%%% CONFIGURATION

% show picture at the top?
\def\showpicture{}

% make images bleak
% run for a in img emoji;do for b in $(ls $a); do echo $a/$b;convert -modulate 130,25,100 $a/$b $a/bleak-$b;done;done
\def\imgpref{bleak-}
%\def\imgpref{}

%%% / CONFIGURATION

%see https://tex.stackexchange.com/questions/11436/automated-age-calculation
\usepackage{datenumber,fp}
\newcounter{dateone}%
\newcounter{datetwo}%

\setmydatenumber{dateone}{1994}{10}{03}%
\setmydatenumber{datetwo}{\the\year}{\the\month}{\the\day}%
\FPsub\result{\thedatetwo}{\thedateone}
\FPdiv\myage{\result}{365.2425}
\newcommand{\mylink}{{\color{gray}\faExternalLink}}
\name{Sergia VOLODIN (Sergei)}
%\contourlength{1pt}
%\ontourumber{10}
\photo{./photo/makeup.jpeg}
\address{\mylink~\href{https://sergia-ch.github.io}{sergia-ch.github.io} / Wehntalerstrasse 20, 8057 Z\"{u}rich, Switzerland \emoji{flag-ch}}
\address{Birth date: 3rd of October 1994 (\FPtrunc\myage{\myage}{0}\myage\,years), Russian \emoji{flag-ru}}
\address{she/her. sergia94 at protonmail dot com \href{http://sergia-ch.github.io}{\emoji{internet}}
	\href{https://twitter.com/sergia_ch}{\imgd{twitter}}
	\href{http://linkedin.com/in/sergeivolodin/}{\imgd{li}} \href{http://github.com/sergia-ch}{\imgd{gh}}
	\emoji{telephone}  +41 78 732 01 34}
%\twocolumn

\begin{document}
	 \renewcommand{\familydefault}{\sfdefault}
	 \sffamily
	
\definecolor{grayitem}{RGB}{80, 80, 80}
\definecolor{pinkitem}{RGB}{100, 0, 100}
\definecolor{headingcolor}{RGB}{50, 0, 100}
\definecolor{graypoint}{RGB}{50, 50, 50}
\definecolor{gray}{RGB}{150, 150, 150}
\definecolor{hrulecolor}{RGB}{25, 0, 150}
\newcommand{\myitem}{\item[\textcolor{pinkitem}{\Huge\raisebox{-2pt}{$\cdot$}}]}

\begin{rSection}{Abstract}

{{\color{pinkitem} \texttt{\$ Sergia:}} artsy recently graduated software engineer and young scientist with interests in open-source democratic and consensual technology with experience in research and development in industry, academia and startups. Looking for an inclusive and diverse team where we all share and discuss what is in our heart, to create something honest, awesome, and balanced to give people what they need.}

\end{rSection}

\begin{rSection}{Education}
\hspace{-1em}

\begin{rSubsection}{\logo{Logo_EPFL.pdf} Swiss Federal Institute of Technology in Lausanne (EPFL) \emoji{flag-ch}}{Master's}{Lausanne, Switzerland}{Sep 2017 -- Apr 2021}
%\item School of Computer and Communication Sciences
\myitem Master's degree in \mybold{Computer Science,} minor in Computational \mybold{Neurosciences}, GPA: \mybold{5.67}/6
\myitem Research Assistant position (2017--2019)
\myitem Thesis \mylink~\href{https://sergia-ch.github.io/causality-simplicity/CauseOccam_Learning_Interpretable_Abstract_Representations_in_Reinforcement_Learning_Environments_via_Model_Sparsity.pdf}{"CauseOccam: Learning Interpretable Abstract Representations in
Reinforcement Learning Environments via Model Sparsity"}
\end{rSubsection}
%\hrule
\begin{rSubsection}{\logo{mipt.png} Moscow Institute of Physics and Technology \emoji{flag-ru}}{Bachelor's}{Moscow, Russia}{June 2017}
%\myitem Department of Control and Applied Mathematics
\item[] Bachelor's degree in \mybold{Applied Mathematics,} GPA: \mybold{4.84}/5
\end{rSubsection}
\end{rSection}

\vspace{-2em}
\begin{rSection}{Skills}
	\vspace{-1em}
	\item \myboldq{Relevant courses:} \mybold{Machine Learning, Software Engineering,} {\small Unsupervised and Reinforcement Learning, Convex Optimization, Distributed Algorithms, Algorithms, Random graph theory, Functional Programming, Set Theory, Random Processes, Functional Analysis, Biological modeling of neural networks, Complexity theory, Learning theory, Neuroscience: behavior and cognition, Neuroprosthetics, Theory and methods for Reinforcement Learning, Optimization for Machine Learning, Computer Vision}
	\item \myboldq{Scientific programming:} \mybold{Keras, TensorFlow, PyTorch,} ray/tune/rllib, tf-agents, scikit-learn, Brian 2, MATLAB, Mathematica, R
	\item \myboldq{Programming languages:} \mybold{Python, C/C++}, TypeScript, Java, {\small Scala, nasm, C\#, AVR C++}
	\item \myboldq{Frameworks:} Qt/QML, Django, {\small Android Studio, OpenGL/GLSL, Unity 3D, Blender, React.js}
	\item \myboldq{Environment:} \mybold{Git, \LaTeX, Bash, Debian/Ubuntu Linux}
	\item \myboldq{Scientific skills:} \mybold{experimental} sections of research papers, working on \mybold{theoretical} problems, scientific presentation, data analysis
	\item \myboldq{Software development:} \mybold{agile} software development (Scrum), CI/CD, debugging, design patterns, concurrent and distributed systems, {\small TCP/IP networking, AVR microcontrollers, Arduino platform, team and project management in \mybold{small startups}}
	\item \myboldq{Languages:} \emoji{flag-us} English: \mylink~\href{https://sergia-ch.github.io/epfl/TOEFL.pdf}{TOEFL iBT \mybold{113}/120}, {\small \emoji{flag-fr} French: A1, \emoji{flag-ru} Russian: native}
\end{rSection}

\begin{rSection}{Research experience}			  
	\begin{rSubsection}{EPFL, Laboratory for Computational Neuroscience \emoji{flag-ch}}{Master's Thesis student}{Lausanne, Switzerland}{Oct 2020 -- Apr 2021}
		\myitem Designed \mylink~\href{https://github.com/sergia-ch/causality-disentanglement-rl}{an algorithm} with \mybold{Python 3, Pytorch and ray} based on the \mylink~"\href{https://arxiv.org/abs/1709.08568}{Consciousness Prior}" proposal that finds a simple causal model of an RL environment in the general case from pixels. The project is a continuation of my Google Research internship (see below)
		\myitem The algorithm works on benchmarks, see \mylink~\href{https://sergia-ch.github.io/causality-simplicity/CauseOccam_Learning_Interpretable_Abstract_Representations_in_Reinforcement_Learning_Environments_via_Model_Sparsity.pdf}{my thesis} for more details
		\myitem The work includes \mybold{theoretical results on abstraction learning as well as a code base with tests and documentation}
		\myitem The thesis defended on the 21st of April 2021 with Adam Gleave (Berkeley/DeepMind) as an external expert
	\end{rSubsection}
	
	\begin{rSubsection}{\logo{chai-logo.png} \logo{UCBerkeley.png} Center for Human-Compatible AI (CHAI), Berkeley \emoji{flag-us}}{Summer Intern}{Berkeley, CA, United States (remote due to COVID-19, from Zurich, Switzerland \emoji{flag-ch})}{June 2020 -- Sep 2020}
		\myitem Designed \mylink~\href{https://github.com/HumanCompatibleAI/better-adversarial-defenses/} {better \mybold{defenses against adversarial policies}} in Multi-Agent Reinforcement Learning via alternating training of opponents using \mybold{Python 3, Tensorflow, ray, rllib}.
		\myitem Ran hyperparameter sweeps on multiple machines with ray and rllib
		\myitem Converted legacy code using stable baselines and Tensorflow 1.0 to rllib and Tensorflow 2.0
		\myitem Results published as a blog post \mylink~"\href{https://forum.effectivealtruism.org/posts/YscrJFofd6S8eJGS8/defending-against-adversarial-policies-in-reinforcement}{Defending against Adversarial Policies in Reinforcement Learning with Alternating Training}" on the Effective Altruism forum
	\end{rSubsection}


	\begin{rSubsection}{{\large\vspace*{-0.7mm}\google}\ Research \emoji{flag-us}}{Software Engineering Intern}{Mountain View, CA, United States}{Nov 2019 -- Feb 2020}
		\myitem Designed an algorithm to uncover a linear \mybold{Causal Model} of a \mybold{Reinforcement Learning} environment using interventions with \mybold{Python 3, Tensorflow, tf-agents}, and analyzed the effect of interventions on the quality of exploration
		\myitem Used TensorFlow and tf-agents to conduct the experiments with large hyperparameter sweeps
		\myitem Results \mylink~\href{https://arxiv.org/abs/2002.05217}{published} as an ICLR CLDM workshop paper
	\end{rSubsection}

	\begin{rSubsection}{EPFL, Computer-Human Interaction in Learning and Instruction laboratory \emoji{flag-ch}}{Research Assistant}{Lausanne, Switzerland}{Sep 2017 -- Aug 2018}
		\myitem Created \mylink~\href{https://github.com/chili-epfl/qml-ar}{a \mybold{library} QML-AR} for seamless \mybold{augmented reality} using \mybold{OpenCV, Qt/C++ and Qt/QML} with competitive performance on Android and small visual negative impact
		\myitem Designed an \mylink~\href{https://youtu.be/B4-2qYsAKH4}{activity for kids} for learning math using AR, tested the application in a classroom setting, analyzed the obtained data
	\end{rSubsection}

	%	\begin{rSubsection}{MIPT, Chair of Data Analysis}{\\Feb 2016 -- Jul 2016}{Research project}{Moscow, Russia}
	%		\myitem Compared machine learning algorithms for the ligand-receptor interaction problem
	%		\myitem Implemented PCC (Probabilistic Classifier Chains) algorithm using scikit-learn library
	%		\myitem The results were published in ITAS proceedings
	%	\end{rSubsection}

%	\begin{rSubsection}{MIPT, Theoretical Mech. dpt.}{Oct 2012 -- Feb 2013}{Technician}{Moscow, Russia}
		%		\myitem Worked on the paper ``Janibekov's effect and the laws of mechanics'' with A.G. Petrov
%		\myitem Designed and implemented numerical simulations for Euler's rotation equations
%		\myitem Checked soundness of the approximation using symbolic computations in Wolfram Mathematica
		%		\myitem The results were published in Doklady Akademii Nauk
%	\end{rSubsection}
\end{rSection}

%\newpage

%\newpage
\vspace{-1em}
\begin{rSection}{Publications}
\vspace{-1em}
\item L\^{e}-Nguy\^{e}n Hoang, Louis Faucon, Aidan Jungo, \mybold{Sergei Volodin,} Dalia Papuc, Orfeas Liossatos, Ben Crulis, Mariame Tighanimine, Isabela Constantin, Anastasiia Kucherenko, Alexandre Maurer, Felix Grimberg, Vlad Nitu, Chris Vossen, Sébastien Rouault, El-Mahdi El-Mhamdi. \mylink~\href{https://arxiv.org/abs/2107.07334}{Tournesol: A quest for a large, secure and trustworthy database of reliable human judgments}, 2021. Code for the platform (backend, ML, frontend), experiments, part of data analysis, writing
\item \logo{iclr.png} \mybold{Sergei Volodin,} Nevan Wichers, Jeremy Nixon. \mylink~\href{https://arxiv.org/abs/2002.05217}{Resolving Spurious Correlations in Causal Models of Environments via Interventions}, 2020. Topic choice, experiments, theory, writing. \mylink~\href{https://causalrlworkshop.github.io/program/cldm_8.html}{ICLR CLDM workshop 2020}.
\item El-M. El-Mhamdi, R. Guerraoui, A. Kucharavy, \mybold{S. Volodin.} \mylink~\href{https://arxiv.org/abs/1902.01686}{The Probabilistic Fault Tolerance of Neural Networks in the Continuous Limit}, 2019. Experiments, theory,  writing. %Under review of {\bf ICML}
\item A. Dymarsky, E. Gryazina, \mybold{S. Volodin}, B. Polyak. \mylink~\href{https://arxiv.org/abs/1810.00896}{Geometry of quadratic maps via convex relaxation}, 2018. Exp-s, theory, writing.% Under review of {\bf SIOPT}
%\item {\bf S. Volodin}, M. Popova, V. Strijov \mylink~\href{http://itas2016.iitp.ru/pdf/1570303389.pdf}{Probabilistic prediction of nuclear receptors’ biological activity}. Proceedings of ITaS, 2016, {\em in Russian}. Implemented the Probabilistic Classifier Chains algorithm using Python and tried it on the dataset
%\todo[inline]{Publish FT on arXiv and update my CV}
\item A. Petrov, \mybold{S. Volodin} \mylink~\href{https://link.springer.com/article/10.1134/S1028335813080041}{Janibekov's effect and the laws of mechanics}. Doklady Akademii Nauk, 2013. Graphics for the article, experiments, first year of my BSc
\end{rSection}

\begin{rSection}{Work experience}
	%Gap year Jan 2015 -- Feb 2016
    \begin{rSubsection}{\mylink~\href{https://faveforfans.com}{Fave For Fans \emoji{fave}}}{Software Engineer}{Platform dedicated to passionate fans, Z\"{u}rich, Switzerland \emoji{flag-ch}}{Sep 2021 -- May 2022}
        \myitem \mybold{Responsible} for the backend development with microservices on Cloudflare Workers with TypeScript, an ArangoDB-based database, automatic data schema validation, and CI/CD with integration tests. First full-time engineer at the company.
        \myitem \mybold{Proposing} and discussing ways to create \mylink~\href{https://sergia-ch.github.io/fave/direct_sharing.png}{more \mybold{ethical and democratic social media} sharing mechanisms}, organizing voting in the team to discuss proposals ("mini \mylink~\href{https://en.wikipedia.org/wiki/Stakeholder_theory}{Stakeholder Capitalism}")
        \myitem \mybold{Research} into ways of obtaining data from third-party services with privacy guarantees
        \myitem \mybold{Conducted} analysis of the database to create better ranking results
    \end{rSubsection}

	\begin{rSubsection}{\mylink~\href{http://tournesol.app}{Tournesol \emoji{sunflower}}}{Co-founder\&ML engineer}{Startup designing better recommender systems, Lausanne, Switzerland \emoji{flag-ch}}{May 2020 -- May 2021}
		\myitem \mybold{Co-founded a startup} working on contributor-driven collaborative recommender systems
		\myitem \mybold{Responsible} for \mylink~\href{https://github.com/tournesol-app/tournesol/tree/old_main}{back-end engineering using \mybold{Django}, and Machine Learning engineering with \mybold{TensorFlow}, the API server}
		\myitem \mybold{Responsible} as well for system administration (Debian), (partially) front-end development with \mybold{React.js} and parts of algorithm design
		\myitem \mybold{Co-authored} \mylink~\href{https://arxiv.org/abs/2107.07334}{the  paper} with our results
	\end{rSubsection}

	\begin{rSubsection}{\mylink~\href{http://escape-control.com}{EscapeControl \emoji{old-key}}}{Founder\&Backend engineer}{Own b2b startup for escape rooms, Moscow, Russia \emoji{flag-ru}}{Jul 2015 -- Feb 2016}
		\myitem \mybold{Created a startup} selling software and hardware for \mylink~\href{https://www.youtube.com/watch?v=kXbubjs7aTA}{real-world escape room games} which allows to speed up the construction and reduce maintenance costs
		\myitem \mybold{Responsible} for back-end software engineering with \mybold{C++/Python}, servers administration, sales and customer support
		\myitem \mybold{Managed} a team of two web developers until a successful launch of the web interface
		\myitem Sold more than forty solutions which are currently running in different countries across the globe and provided remote support
	\end{rSubsection}

	%\begin{rSubsection}{\href{http://phobia.ru}{Claustrophobia}}{July 2014 -- Feb 2015}{Developer}{Moscow, Russia}
	%	\myitem Created system architecture for the real-world escape room game
	%	\myitem Implemented the solution using C++ (Atmel AVR, Linux)
	%	\myitem Results description: \url{habr.ru/p/258585/} (in Russian)
	%\end{rSubsection}

\end{rSection}

\dotfill

\newpage
\dotfill
\begin{center}
	\mybold{\em EXTRA}
\end{center}
%\changefontsizes{7pt}

\begin{rSection}{Extra Research Experience}
	\begin{rSubsection}{EPFL, Distributed Computing Laboratory \emoji{flag-ch}}{Research Assistant}{Lausanne, Switzerland}{Sep 2018 -- Oct 2019}
		\myitem Investigated \mybold{fault tolerance} of a neural network using \mybold{Taylor approximation}
		\myitem Introduced the {\em continuous limit} to \mylink~\href{https://arxiv.org/abs/1902.01686}{bound the error}, and compared to the Neural Tangent Kernel limit case
		\myitem Conducted \mylink~\href{https://github.com/LPD-EPFL/ProbabilisticFaultToleranceNNs}{experiments} to test the theory using \mybold{Keras} including the \mybold{implementation} of custom layers and regularizers
		%		\myitem Wrote and submitted a paper to ICML: \mylink~\href{https://arxiv.org/abs/1902.01686}{arXiv:1902.01686}
	\end{rSubsection}

	\begin{rSubsection}{Skolkovo Institute of Science and Technology, Center for Energy Systems \emoji{flag-ru}}{Research Intern}{Moscow, Russia}{Sep 2016 -- Jul 2017}
	%		\myitem Worked on the Power Flow Feasibility problem with Prof. Anatoly Dymarsky, Dr. Elena Gryazina, and Prof. Boris T. Polyak
	\myitem Characterized using \mybold{numerical optimization} and \mybold{theoretically} the structure of the set of boundary non-convexities of an image of a quadratic map in case the number of non-convexities is infinite
	\myitem Designed and implemented \mylink~\href{https://github.com/sergeivolodin/CAQM}{the Convexity Analysis of Quadratic Maps \mybold{library}} using \mybold{MATLAB} which gives approximate solutions to a number of problems involving quadratic maps
   \end{rSubsection}
\end{rSection}

\begin{rSection}{Extra Work Experience}
	\begin{rSubsection}{ITBrat \emoji{stocks}}{Software Engineer}{Algorithmic trading startup, Moscow, Russia \emoji{flag-ru}}{Jul 2015 -- Feb 2016}
	\myitem \mybold{Developed} algorithmic trading application from initial discussion with the team to deployment and supporting in \mybold{C++}
	\myitem Added low-level user-space networking to the project which allowed to decrease latency and increase profit
	\myitem \mybold{Responsible} for the performance of the code
	%		\myitem Designed and supported the environment for the algorithm: build stage, version control, performance analysis using network dumps
	\end{rSubsection}
\end{rSection}

\begin{rSection}{Research interests}
	Artificial Intelligence Safety/Ethics, Artificial Intelligence, Machine Learning, Causal Reasoning, Neuroscience, Adversarial policies, Mathematical Optimization, Robotics, Consciousness research %, Artificial General Intelligence, Consciousness
\end{rSection}

\begin{rSection}{Scholarships}
	\vspace{-1em}
	\item \mylink~\href{https://ic.epfl.ch/ResearchScholars}{Research Scholars}, a paid \mybold{Research Assistant} position, Swiss Federal Institute of Technology in Lausanne (EPFL), 2017 -- 2019
	\item \mylink~\href{http://phystech-foundation.org/}{Abramov Fund's} scholarship for excellent \mybold{grades,} 2014
\end{rSection}

\begin{rSection}{Projects}
%		\begin{rSubsection}{Learning Interpretable Abstract Representations in Reinforcement Learning via Model Sparsity}{2020}{EPFL semester/thesis project, advised by Dr. Johanni Brea and Prof. Wulfram Gerstner \emoji{flag-ch}}{}
%		\myitem \mylink~\href{https://www.overleaf.com/read/nqgjrjbcybrp}{Designed an algorithm} to learn {\bf Abstract Representations} for {\bf Causal Models} of {\bf Reinforcement Learning} environments via {\bf Model Sparsity Constraint}
%		\myitem \mylink~\href{https://github.com/sergeivolodin/causality-disentanglement-rl}{Implemented the proposed algorithm in Tensorflow/pytorch} and tested it on a proof-of-concept setting
%		\myitem \mylink~\href{https://www.youtube.com/watch?v=4cO4XU15isQ}{Presented the project} at the MLSS 2020 summer school (remote due to COVID-19)
%        \myitem \mylink~\href{https://github.com/sergeivolodin/ms-thesis/releases/download/v1.0/thesis_volodin.pdf}{Extended} the framework to non-linear environments and grid worlds
%	\end{rSubsection}

	\begin{rSubsection}{Safe Proximal Policy Optimization}{}{EPFL EE-618 course project, advised by Dr. Kamalaruban Parameswaran and Prof. Volkan Cevher, Lausanne, Switzerland \emoji{flag-ch}}{2019}
		\myitem \mylink~\href{https://www.overleaf.com/read/cvxkswbspgpb}{Added a projection step} to the \mybold{Proximal Policy Optimization} algorithm to comply with requirements of \mybold{Constrained Markov Decision Processes}
		\myitem \mylink~\href{https://github.com/sergeivolodin/SafeContinuousStateRL}{Implemented code in Tensorflow} and tested it in simple environments
		\myitem Presented the project at the RLSS 2019 summer school (Lille, France)
	\end{rSubsection}

	\begin{rSubsection}{Quadcopter drone from scratch project}{}{Russia \emoji{flag-ru}}{2012 -- 2014}
		\myitem Developed \mylink~\href{https://github.com/it-workshop/Quadrocopter}{an algorithm} in C++ for stabilization of a quadcopter drone from scratch using AVR microcontrollers, IMU sensors and PID regulators
		\myitem \mybold{Co-managed} the project consisting of 2-5 developers
		\myitem Conducted the analysis of launches to improve \mylink~\href{https://www.youtube.com/watch?v=AxDoO-gNRtc}{flying quality}
		\myitem Results were \mybold{published} as a \mylink~\href{http://web.archive.org/web/20141016114551/http://habrahabr.ru/company/technoworks/blog/216437/}{popular science article} {\em (in Russian)}
	\end{rSubsection}
\end{rSection}

\begin{rSection}{Conferences and summer schools}
%\todo[inline]{Choose only the best ones}
\vspace{-1em}
\item \emoji{flag-de} \mylink~\href{http://mlss.tuebingen.mpg.de/2020/}{Machine Learning Summer School, 2020} (virtual due to COVID-19){, \em \mybold{poster} presenter}
%\item \mylink~\href{https://appliedmldays.org}{Applied Machine Learning Days} (Lausanne, Switzerland, 2019){, \em participant of workshops}
\item \emoji{flag-fr} \mylink~\href{https://rlss.inria.fr}{Reinforcement Learning Summer School, 2019} (Lille, France){, \em \mybold{poster} presenter, selected to receive \mybold{financial help}}
\item \emoji{flag-fr} \mylink~\href{https://ds3-datascience-polytechnique.fr}{Data science summer school, 2019} (Paris, France){, \em \mybold{poster} presenter}
\item \emoji{flag-it} \mylink~\href{https://www.qtday.it/agenda/session/52811}{QtDay 2019} (Firenze, Italy){, \em \mybold{speaker,} one hour session on qml-ar}
\item \emoji{flag-fr} \mylink~\href{https://project.inria.fr/paiss/}{P.A.I.S.S.} AI Summer School, INRIA Grenoble, 2018{, \em participant in tutorials given by top experts; \mylink~\href{http://www.europe.naverlabs.com/Blog/Students-at-PAISS}{selected} to receive financial aid }
%\item \mylink~\href{http://deepbayes.ru}{DeepBayes} school on Bayesian methods in Deep Learning (Moscow, 2017){, \em participant of lectures and practical sessions on Bayesian Methods}
\item \emoji{flag-ru} \mylink~\href{http://iitp.ru/en/conferences/itas}{Information Technologies and Systems} (Saint-Petersburg, Repino, 2016){, \em \mybold{speaker,} poster presenter}
%\item \mylink~\href{https://sites.google.com/site/traditionalschool/about}{School} ``Control, Information, Optimization'' (Saint-Petersburg, Repino, 2016){, \em Poster presenter}
\end{rSection}

%\begin{rSection}{Courses}
%	\item Getting Started with Deep Learning (NVIDIA Deep Learning Workshop, MIPT, Feb 2017)
%	\item Approaches to Object Detection using DIGITS (NVIDIA Deep Learning Workshop, MIPT, Feb 2017)
%\end{rSection}

\begin{rSection}{Competitions}
\vspace{-1em}
\item \emoji{flag-ch} \mylink~\href{https://hashcodejudge.withgoogle.com/scoreboard}{Google HashCode} Qualification round coding contest, \mybold{top 6\%} (team EPFL\_Noobs), managed the team, developed algorithms and did the coding, 2019\vspace{1em}\\
\emoji{flag-ru}
\mylink~\href{http://web.archive.org/web/20170224094223/http://rl.deephack.me/}{DeepHack.RL} hackathon on Deep \mybold{Reinforcement} Learning for Atari games, managed the team and developed an \mylink~\href{https://github.com/sergeivolodin/deephack.rl}{evolutionary algorithm with an autoencoder}, MIPT, Moscow, Russia, 2017
\end{rSection}

\begin{rSection}{Interests}
	Music, Dancing, Running ($1/2$ marathon 2018), Snowboarding, Swimming, Philosophy, DIY, Activism
	% Effective Altruism -- what's going on with it these days???.. :surprised:
\end{rSection}

\begin{rSection}{Volunteering}
\begin{rSubsection}{Better Russia}{}{\mylink~\href{https://betterrussia.org}{Community} of pro-democracy Russians, Zurich, Switzerland \emoji{flag-ch}}{2022}
	\item[] Making posters for the demonstrations, organizing voting on important decisions among members, brainstorming on the situation to find a way forward
\end{rSubsection}
	
	
\begin{rSubsection}{Effective Altruism Lausanne}{}{Local \mylink~\href{https://effectivealtruism.org}{EA} community, Lausanne, Switzerland \emoji{flag-ch}}{2019}
	\item[] Co-founding the group, \mylink~\href{http://eageneva.org/about#our-members}{introduction workshop speaker}, running a \mylink~\href{https://docs.google.com/document/d/1prejPACr08nUVztHRBc4nj4upKtDc9_EJOYi1alWuYM/edit?usp=sharing}{discussion group} on AI safety and theory, newsletter management and writing, Facebook events announcements, managing open discussions
\end{rSubsection}

\begin{rSubsection}{Artificial Intelligence Governance Forum}{}{\mylink~\href{https://ai-gf.com/}{AI governance conference}, Geneva, Switzerland (2019), virtual due to COVID-19 (2020) \emoji{flag-ch}}{2019, 2020}
	\item[] Time-keeping, technical support, small tutorial on neural networks
\end{rSubsection}

\begin{rSubsection}{Applied Machine Learning Days}{}{Machine learning \mylink~\href{https://www.appliedmldays.org/}{conference}, Lausanne, Switzerland \emoji{flag-ch}}{2019}
\item[]	Technical help for presenters, badge check
\end{rSubsection}
	\begin{rSubsection}{Anti-corruption foundation (A. Navalny)}{}{A \mylink~\href{https://en.wikipedia.org/wiki/Anti-Corruption_Foundation}{non-profit} aimed at investigating corruption, Moscow, Russia \emoji{flag-ru}}{2017}
\item[] Conveyed the results of the investigations by talking to people on the streets as a volunteer
	\end{rSubsection}
\end{rSection}

% failed applications for keeping the Balance of the Force
% OpenAI (2017, 2018)
% Anti-corruption foundation hackathon 2017
% Google Software Dev (2016, 2017, 2018, 2019)
% Google Research (2018, 2019)
% Facebook Software Dev 2017
% Facebook dinner 2018
% Microsoft Research 2017
% Samsung Research 2017
% Hyperloop EPFL Software Dev 2018
% LauzHack 2018
% Effective Altruism Global 2017
% 80000 career counceling 2017
% Research Scholars MLO 2017
% Summer@EPFL 2017
% Summer OIST 2017
% Summer ETHZ 2017
% Internship ELCA 2017
% Failed GRE Subject math, GRE
% Missed TOEFL
% ICML paper rejection
% SIOPT paper rejection
% MIRI MSFP didn't go because of depression
% Human-aligned summer school didn't go because of depression

\end{document}
