\documentclass{resume} % Use the custom resume.cls style
%\documentclass{twocolumn} % Use the custom resume.cls style
\usepackage{fontawesome,graphicx}
\usepackage{hyperref}
\usepackage{todonotes}
\usepackage{amsmath}
\usepackage[left=0.2in,top=0.2in,right=0.2in,bottom=0.2in]{geometry} % Document margins
\hypersetup{
	colorlinks=true,
	urlcolor=[rgb]{0.3, 0.3, 0.3},
}

\newcommand*{\img}[1]{%
	\raisebox{-.02\baselineskip}{%
		\includegraphics[
		height=\baselineskip,
		width=\baselineskip,
		keepaspectratio,
		]{#1}%
	}%
}

%see https://tex.stackexchange.com/questions/11436/automated-age-calculation
\usepackage{datenumber,fp}
\newcounter{dateone}%
\newcounter{datetwo}%

\setmydatenumber{dateone}{1994}{10}{03}%
\setmydatenumber{datetwo}{\the\year}{\the\month}{\the\day}%
\FPsub\result{\thedatetwo}{\thedateone}
\FPdiv\myage{\result}{365.2425}
\newcommand{\mylink}{{\color{gray}\faExternalLink}}
\name{Sergei VOLODIN}
\address{Route de la Chocolati\`ere 29 A / 009, \'Echandens, Switzerland}
\address{Birth date: 3rd October 1994 (\FPtrunc\myage{\myage}{0}\myage\,years), Russian}
\address{sergei.volodin@epfl.ch \href{http://sergeivolodin.github.io}{\img{./img/www.png}}
	\href{http://linkedin.com/in/sergeivolodin/}{\img{./img/li.png}} \href{http://github.com/sergeivolodin}{\img{./img/gh.png}} +41 78 732 01 34}
\twocolumn

\begin{document}
\definecolor{grayitem}{RGB}{80, 80, 80}
\definecolor{grayheading}{RGB}{80, 80, 80}
\definecolor{graypoint}{RGB}{50, 50, 50}
\definecolor{gray}{RGB}{150, 150, 150}
\newcommand{\myitem}{\item[\textcolor{grayitem}{$\cdot$}]}
\begin{rSection}{Education}
\hspace{-1em}
\begin{rSubsection}{Swiss Federal Institute of Technology in Lausanne (EPFL)}{}{Lausanne, Switzerland}{Sep 2017 -- 2021}
%\item School of Computer and Communication Sciences
\myitem Master's degree in {\bf Computer Science,} GPA: {\bf 5.68}/6
\myitem Minor in Computational {\bf Neurosciences}
\myitem Research Assistant position (2017--)
\end{rSubsection}
%\hrule
\begin{rSubsection}{Moscow Institute of Physics and Technology}{}{Moscow, Russia}{June 2017}
%\myitem Department of Control and Applied Mathematics
\item[] Bachelor's degree in {\bf Applied Mathematics,} GPA: {\bf 4.84}/5
\end{rSubsection}
\end{rSection}

\begin{rSection}{Skills}
	\vspace{-1em}
	\item {\color{grayheading}\bf Relevant courses:} {\bf Machine Learning, Software Engineering,} {\small Unsupervised and Reinforcement Learning, Convex Optimization, Distributed Algorithms, Algorithms, Random graph theory, Functional Programming, Set Theory, Random Processes, Functional Analysis, Biological modeling of neural networks, Complexity theory, Learning theory, Neuroscience: behavior and cognition, Neuroprosthetics, Theory and methods for Reinforcement Learning, Optimization for Machine Learning}
	\item {\color{grayheading}\bf Scientific programming:} {\bf Keras, TensorFlow,} Theano, scikit-learn, PyTorch, Brian 2, MATLAB, Mathematica, R
	\item {\color{grayheading}\bf Programming languages:} Python, C/C++, Java, {\small Scala, nasm, C\#, AVR C++}
	\item {\color{grayheading}\bf Frameworks:} Qt/QML, Django, {\small Android Studio, OpenGL/GLSL, Unity 3D, Blender}
	\item {\color{grayheading}\bf Environment:} Git, \LaTeX, Bash, Debian/Ubuntu Linux
	\item {\color{grayheading}\bf Scientific skills:} {\bf experimental} sections of research papers, working on {\bf theoretical} problems, scientific presentation, data analysis
	\item {\color{grayheading}\bf Software development:} team and project {\bf management,} agile software development (Scrum), debugging, design patterns, concurrent and distributed systems, {\small TCP/IP networking, AVR microcontrollers, Arduino platform}
	\item {\color{grayheading}\bf Languages:} English: TOEFL iBT {\bf 112}/120, {\small French: A1, Russian: native}
\end{rSection}

\begin{rSection}{Research experience}

	\begin{rSubsection}{EPFL, Distributed Computing Laboratory}{Research Assistant}{Lausanne, Switzerland}{Sep 2018 -- present}
		\myitem Investigated fault tolerance of a neural network using {\bf Taylor approximation}
		\myitem Conducted experiments to test the theory using Keras including the {\bf implementation} of custom layers and regularizers
%		\myitem Wrote and submitted a paper to ICML: \mylink~\href{https://arxiv.org/abs/1902.01686}{arXiv:1902.01686}
	\end{rSubsection}

	\begin{rSubsection}{EPFL, Computer-Human Interaction in Learning and Instruction laboratory}{Research Assistant}{Lausanne, Switzerland}{Sep 2017 -- Aug 2018}
		\myitem Created \mylink~\href{https://github.com/chili-epfl/qml-ar}{a {\color{black} \bf library} QML-AR} for seamless {\bf augmented reality} using OpenCV with competitive performance on Android and small visual negative impact
		\myitem Designed an \mylink~\href{https://youtu.be/B4-2qYsAKH4}{activity for kids} for learning math using AR, tested the application in a classroom setting, analyzed the obtained data
	\end{rSubsection}
	
	\begin{rSubsection}{Skolkovo Institute of Science and Technology, \\Center for Energy Systems}{Research Intern}{Moscow, Russia}{Sep 2016 -- Jul 2017}
		%		\myitem Worked on the Power Flow Feasibility problem with Prof. Anatoly Dymarsky, Dr. Elena Gryazina, and Prof. Boris T. Polyak
		\myitem Characterized using {\bf numerical optimization} and {\bf theoretically} the structure of the set of boundary non-convexities of an image of a quadratic map in case the number of non-convexities is infinite
		\myitem Designed and implemented \mylink~\href{https://github.com/sergeivolodin/CAQM}{the Convexity Analysis of Quadratic Maps {\bf \color{black} library}} which gives approximate solutions to a number of problems involving quadratic maps
	\end{rSubsection}
	
	%	\begin{rSubsection}{MIPT, Chair of Data Analysis}{\\Feb 2016 -- Jul 2016}{Research project}{Moscow, Russia}
	%		\myitem Compared machine learning algorithms for the ligand-receptor interaction problem
	%		\myitem Implemented PCC (Probabilistic Classifier Chains) algorithm using scikit-learn library
	%		\myitem The results were published in ITAS proceedings
	%	\end{rSubsection}
	
%	\begin{rSubsection}{MIPT, Theoretical Mech. dpt.}{Oct 2012 -- Feb 2013}{Technician}{Moscow, Russia}
		%		\myitem Worked on the paper ``Janibekov's effect and the laws of mechanics'' with A.G. Petrov
%		\myitem Designed and implemented numerical simulations for Euler's rotation equations
%		\myitem Checked soundness of the approximation using symbolic computations in Wolfram Mathematica
		%		\myitem The results were published in Doklady Akademii Nauk
%	\end{rSubsection}
\end{rSection}

\begin{rSection}{Research interests}
	Artificial Intelligence, Machine Learning, Artificial Intelligence Safety, Mathematical Optimization, Robotics
\end{rSection}

\begin{rSection}{Scholarships}
	\vspace{-1em}
	\item \mylink~\href{https://ic.epfl.ch/ResearchScholars}{Research Scholars}, a paid {\bf Research Assistant} position, Swiss Federal Institute of Technology in Lausanne (EPFL), 2017 -- 2019
	\item \mylink~\href{http://phystech-foundation.org/}{Abramov Fund's} scholarship for excellent {\bf grades,} 2014
\end{rSection}

\newpage

\begin{rSection}{Publications}
\vspace{-1em}
%\item El Mahdi El Mhamdi, R. Guerraoui, {\bf \color{grayheading} S. Volodin.} \mylink~\href{https://arxiv.org/abs/1902.01686}{Fatal Brain Damage}, 2019. Experiments, theory,  writing. %Under review of {\bf ICML}
\item A. Dymarsky, E. Gryazina, B. Polyak, {\bf \color{grayheading} S. Volodin.} \mylink~\href{https://arxiv.org/pdf/1810.00896.pdf}{Geometry of quadratic maps via convex relaxation}, 2018. Experiments, theory, writing.% Under review of {\bf SIOPT}
%\item {\bf S. Volodin}, M. Popova, V. Strijov \mylink~\href{http://itas2016.iitp.ru/pdf/1570303389.pdf}{Probabilistic prediction of nuclear receptors’ biological activity}. Proceedings of ITaS, 2016, {\em in Russian}. Implemented the Probabilistic Classifier Chains algorithm using Python and tried it on the dataset
\todo[inline]{Publish FT on arXiv and update my CV}
\item A. Petrov, {\bf \color{grayheading} S. Volodin} \mylink~\href{https://link.springer.com/article/10.1134/S1028335813080041}{Janibekov's effect and the laws of mechanics}. Doklady Akademii Nauk, 2013. Helped to create graphics for the article and provided experimental section during the {\bf first year} of my BSc degree at MIPT
\end{rSection}

\begin{rSection}{Work experience}
	%Gap year Jan 2015 -- Feb 2016
	
	\begin{rSubsection}{\mylink~\href{http://escape-control.com}{EscapeControl}}{Jul 2015 -- Feb 2016}{Own b2b startup for escape rooms, Moscow, Russia}{}
		\myitem {\bf Created a startup} selling software and hardware for real-world escape room games which allows to speed up the construction and reduce maintenance costs
		\myitem {\bf Responsible} for back-end software engineering, servers administration, sales and customer support
		\myitem {\bf Managed} a team of two web developers until a successful launch of the web interface
		\myitem Sold more than twenty solutions which are currently running in different countries across the globe and provided remote support
	\end{rSubsection}
	
	\begin{rSubsection}{ITBrat}{Jul 2015 -- Feb 2016}{Algorithmic trading startup, Moscow, Russia}{}
		\myitem {\bf Developed} algorithmic trading application from initial discussion with the team to deployment and supporting
		\myitem Added low-level user-space networking to the project which allowed to decrease latency and increase profit
		\myitem {\bf Responsible} for the performance of the code
%		\myitem Designed and supported the environment for the algorithm: build stage, version control, performance analysis using network dumps
	\end{rSubsection}
	
	%\begin{rSubsection}{\href{http://phobia.ru}{Claustrophobia}}{July 2014 -- Feb 2015}{Developer}{Moscow, Russia}
	%	\myitem Created system architecture for the real-world escape room game
	%	\myitem Implemented the solution using C++ (Atmel AVR, Linux)
	%	\myitem Results description: \url{habr.ru/p/258585/} (in Russian)
	%\end{rSubsection}
	
\end{rSection}

\begin{rSection}{Projects}
	\begin{rSubsection}{Quadcopter drone from scratch project}{2012 -- 2014}{}{}
		\myitem Developed \mylink~\href{https://github.com/it-workshop/Quadrocopter}{an algorithm} in C++ for stabilization of a quadcopter drone from scratch using AVR microcontrollers, IMU sensors and PID regulators
		\myitem {\bf Managed} the project consisting of 2-5 developers
		\myitem Conducted the analysis of launches to improve flying quality
		\myitem Results were {\bf published} as a \mylink~\href{http://web.archive.org/web/20141016114551/http://habrahabr.ru/company/technoworks/blog/216437/}{popular science article} {\em (in Russian)}
	\end{rSubsection}
\end{rSection}

\begin{rSection}{Conferences and summer schools}
\todo[inline]{Choose only the best ones}
\vspace{-1em}
%\item \mylink~\href{https://appliedmldays.org}{Applied Machine Learning Days} (Lausanne, Switzerland, 2019){, \em participant of workshops}
\item \mylink~\href{https://rlss.inria.fr}{Reinforcement Learning Summer School, 2019} (Lille, France){, \em poster presenter, selected to receive financial help}
\item \mylink~\href{https://ds3-datascience-polytechnique.fr}{Data science summer school} (Paris, France){, \em poster presenter}
\item \mylink~\href{https://www.qtday.it/agenda/session/52811}{QtDay 2019} (Firenze, Italy){, \em speaker}
\item \mylink~\href{https://project.inria.fr/paiss/}{P.A.I.S.S.} (AI Summer School) (INRIA Grenoble, 2018){, \em participant in tutorials given by top experts; \mylink~\href{http://www.europe.naverlabs.com/Blog/Students-at-PAISS}{selected} to receive financial help }
\item \mylink~\href{http://deepbayes.ru}{DeepBayes} school on Bayesian methods in Deep Learning (Moscow, 2017){, \em participant of lectures and practical sessions on Bayesian Methods}
\item \mylink~\href{http://iitp.ru/en/conferences/itas}{Information Technologies and Systems} (Saint-Petersburg, Repino, 2016){, \em {\bf speaker,} poster presenter}
%\item \mylink~\href{https://sites.google.com/site/traditionalschool/about}{School} ``Control, Information, Optimization'' (Saint-Petersburg, Repino, 2016){, \em Poster presenter}
\end{rSection}

%\begin{rSection}{Courses}
%	\item Getting Started with Deep Learning (NVIDIA Deep Learning Workshop, MIPT, Feb 2017)
%	\item Approaches to Object Detection using DIGITS (NVIDIA Deep Learning Workshop, MIPT, Feb 2017)
%\end{rSection}

\begin{rSection}{Competitions}
\vspace{-1em}
\item \mylink~\href{https://hashcodejudge.withgoogle.com/scoreboard}{Google HashCode} Qualification round coding contest, {\bf top 6\%} (team EPFL\_Noobs), managed the team, developed algorithms and did the coding, 2019\vspace{1em}\\
\mylink~\href{http://web.archive.org/web/20170224094223/http://rl.deephack.me/}{DeepHack.RL} hackathon on Deep {\bf Reinforcement} Learning for Atari games, managed the team and developed an \mylink~\href{https://github.com/sergeivolodin/deephack.rl}{evolutionary algorithm with an autoencoder}, MIPT, Moscow, Russia, 2017
\end{rSection}

\begin{rSection}{Interests}
	{\bf Effective Altruism,} Philosophy, Running ($1/2$ marathon 2018), Snowboarding, Swimming, Dancing Rock'n'Roll
\end{rSection}

\begin{rSection}{Volunteering}
\begin{rSubsection}{Effective Altruism Lausanne}{2019}{Local \mylink~\href{https://effectivealtruism.org}{EA} community}{Lausanne, Switzerland}
	\item[]	Co-founding the group, introduction workshop speaker, newsletter management and writing, Facebook events announcements, managing open discussions
\end{rSubsection}
\begin{rSubsection}{Applied Machine Learning Days}{2019}{Machine learning \mylink~\href{https://www.appliedmldays.org/}{conference}}{Lausanne, Switzerland}
\item[]	Technical help for presenters, badge check
\end{rSubsection}
	\begin{rSubsection}{Anti-corruption foundation}{2017}{A \mylink~\href{https://en.wikipedia.org/wiki/Anti-Corruption_Foundation}{non-profit} aimed at investigating corruption}{Moscow, Russia}
\item[] Conveyed the results of the investigations by talking to people on the streets as a volunteer
	\end{rSubsection}
\end{rSection}

% failed applications for keeping the Balance of the Force
% OpenAI (2017, 2018)
% Anti-corruption foundation hackathon 2017
% Google Software Dev (2016, 2017, 2018, 2019)
% Google Research (2018, 2019)
% Facebook Software Dev 2017
% Facebook dinner 2018
% Microsoft Research 2017
% Samsung Research 2017
% Hyperloop EPFL Software Dev 2018
% LauzHack 2018
% Effective Altruism Global 2017
% 80000 career counceling 2017
% Research Scholars MLO 2017
% Summer@EPFL 2017
% Summer OIST 2017
% Summer ETHZ 2017
% Internship ELCA 2017
% Failed GRE Subject math, GRE
% Missed TOEFL
% ICML paper rejection
% SIOPT paper rejection

\end{document}
