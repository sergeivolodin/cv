\documentclass{resume} % Use the custom resume.cls style
%\documentclass{twocolumn} % Use the custom resume.cls style
\usepackage{fontawesome,graphicx}
\usepackage{hyperref}
\usepackage[left=0.5in,top=0.5in,right=0.5in,bottom=0.5in]{geometry} % Document margins

\hypersetup{
	colorlinks=true,
	urlcolor=[rgb]{0.3, 0.3, 0.3},
}

\newcommand*{\img}[1]{%
	\raisebox{-.02\baselineskip}{%
		\includegraphics[
		height=\baselineskip,
		width=\baselineskip,
		keepaspectratio,
		]{#1}%
	}%
}

%see https://tex.stackexchange.com/questions/11436/automated-age-calculation
\usepackage{datenumber,fp}
\newcounter{dateone}%
\newcounter{datetwo}%

\setmydatenumber{dateone}{1994}{10}{03}%
\setmydatenumber{datetwo}{\the\year}{\the\month}{\the\day}%
\FPsub\result{\thedatetwo}{\thedateone}
\FPdiv\myage{\result}{365.2425}

\name{Sergei VOLODIN}
\address{Route de la Chocolati\`ere 29 A / 009, \'Echandens, Switzerland}
\address{Birth date: 3rd October 1994 (\FPtrunc\myage{\myage}{0}\myage\,years), Russian}
\address{sergei.volodin@epfl.ch \href{http://sergeivolodin.github.io}{\img{./img/www.png}}
	\href{http://linkedin.com/in/sergeivolodin/}{\img{./img/li.png}} \href{http://github.com/sergeivolodin}{\img{./img/gh.png}} +41 78 732 01 34}
\twocolumn
\begin{document}

\begin{rSection}{Education}
\hspace{-1em}
\begin{rSubsection}{\bf Swiss Federal Institute of Technology in Lausanne (EPFL)}{}{Lausanne, Switzerland}{Sep 2017 -- June 2020}
%\item School of Computer and Communication Sciences
\item Master's degree in Computer Science
\item Minor in Computational Neurosciences
\item GPA: {\bf 5.61}/6
\end{rSubsection}

\begin{rSubsection}{\bf Moscow Institute of Physics and Technology}{}{Moscow, Russia}{June 2017}
%\item Department of Control and Applied Mathematics
\item Bachelor's degree im Applied Mathematics
\item GPA: {\bf 4.84}/5
\end{rSubsection}
\end{rSection}

\begin{rSection}{Skills}
	\vspace{-1em}
	\item {\bf Relevant courses:} Machine Learning, Software Engineering, {\small Unsupervised and Reinforcement Learning, Convex Optimization, Distributed Algorithms, Algorithms, Random graph theory, Functional Programming, Set Theory, Random Processes, Functional Analysis, Biological modeling of neural networks, Complexity theory}
	\item {\bf Scientific programming:} Keras, TensorFlow, Theano, scikit-learn, Brian 2, MATLAB, Mathematica, R
	\item {\bf Programming languages:} C/C++, Python, AVR C++, Scala, Java, nasm, C\#
	\item {\bf Frameworks:} Qt/QML, Django, Android Studio, OpenGL/GLSL, Unity 3D, Blender
	\item {\bf Environment:} Git, \LaTeX, Bash, Debian/Ubuntu Linux
	\item {\bf Scientific skills:} experimental sections of research papers, working on theoretical problems, scientific presentation, data analysis
	\item {\bf Software development:} team and project management, agile software development (Scrum), debugging, TCP/IP networking, design patterns, concurrent and distributed systems, AVR microcontrollers, Arduino platform
	\item {\bf Languages:} English (TOEFL iBT 112/120), French (beginner), Russian (native)
\end{rSection}

\begin{rSection}{Research experience}

	\begin{rSubsection}{Swiss Federal Institute of Technology in Lausanne (EPFL), Distributed Computing Laboratory}{Research Assistant}{Lausanne, Switzerland}{Sep 2018 -- present}
		\item Improved the probabilistic bound on the error of a neural network in case of independent neuron failures
		\item Conducted experiments to test the improved theory using Keras and Tensorflow including the implementation of custom layers and regularizers
	\end{rSubsection}

	\begin{rSubsection}{EPFL, Computer-Human Interaction in Learning and Instruction laboratory}{Research Assistant}{Lausanne, Switzerland}{Sep 2017 -- Aug 2018}
		\item Created \faExternalLink~\href{https://github.com/chili-epfl/qml-ar}{a library QML-AR} for seamless augmented reality using OpenCV with competitive performance on Android and small visual negative impact
		\item Designed an activity for kids for learning math using AR, tested the application in a classroom setting, analyzed the obtained data
	\end{rSubsection}
	
	\begin{rSubsection}{Skolkovo Institute of Science and Technology, Center for Energy Systems}{Research Intern}{Moscow, Russia}{Sep 2016 -- Jul 2017}
		%		\item Worked on the Power Flow Feasibility problem with Prof. Anatoly Dymarsky, Dr. Elena Gryazina, and Prof. Boris T. Polyak
		\item Characterized using numerical optimization and theoretically the structure of the set of boundary non-convexities of an image of a quadratic map in case the number of non-convexities is infinite
		\item Designed and implemented \faExternalLink~\href{https://github.com/sergeivolodin/CAQM}{the Convexity Analysis of Quadratic Maps library} which gives approximate solutions to a number of problems involving quadratic maps
	\end{rSubsection}
	
	%	\begin{rSubsection}{MIPT, Chair of Data Analysis}{\\Feb 2016 -- Jul 2016}{Research project}{Moscow, Russia}
	%		\item Compared machine learning algorithms for the ligand-receptor interaction problem
	%		\item Implemented PCC (Probabilistic Classifier Chains) algorithm using scikit-learn library
	%		\item The results were published in ITAS proceedings
	%	\end{rSubsection}
	
%	\begin{rSubsection}{MIPT, Theoretical Mech. dpt.}{Oct 2012 -- Feb 2013}{Technician}{Moscow, Russia}
		%		\item Worked on the paper ``Janibekov's effect and the laws of mechanics'' with A.G. Petrov
%		\item Designed and implemented numerical simulations for Euler's rotation equations
%		\item Checked soundness of the approximation using symbolic computations in Wolfram Mathematica
		%		\item The results were published in Doklady Akademii Nauk
%	\end{rSubsection}
\end{rSection}

\begin{rSection}{Research interests}
	Artificial Intelligence, Machine Learning, Artificial Intelligence Safety, Mathematical Optimization, Robotics
\end{rSection}

\begin{rSection}{Publications}
\vspace{-1em}
\item A. Dymarsky, E. Gryazina, {\bf S. Volodin,} B. Polyak. \faExternalLink~\href{https://arxiv.org/pdf/1810.00896.pdf}{Geometry of quadratic maps via convex relaxation}. arXiv:1810.00896, 2018. Experimental section, theoretical derivations, editing
\item {\bf S. Volodin}, M. Popova, V. Strijov \faExternalLink~\href{http://itas2016.iitp.ru/pdf/1570303389.pdf}{Probabilistic prediction of nuclear receptors’ biological activity}. Proceedings of ITaS, 2016, {\em in Russian}. Implemented the Probabilistic Classifier Chains algorithm using Python and tried it on the dataset
\item A. Petrov, {\bf S. Volodin} \faExternalLink~\href{https://link.springer.com/article/10.1134/S1028335813080041}{Janibekov's effect and the laws of mechanics}. Doklady Akademii Nauk, 2013. Helped to create graphics for the article and provided experimental section during my first year at MIPT
\end{rSection}

\begin{rSection}{Work experience}
	%Gap year Jan 2015 -- Feb 2016
	
	\begin{rSubsection}{\faExternalLink~\href{http://escape-control.com}{EscapeControl}}{Jul 2015 -- Feb 2016}{Own b2b startup for escape rooms, Moscow, Russia}{}
		\item Created a startup selling software and hardware for real-world escape room games which allows to speed up the construction and reduce maintenance costs
		\item Responsible for back-end software engineering, servers administration, sales and customer support
		\item Managed a team of two web developers until a successful launch of the web interface
		\item Sold more than twenty solutions which are currently running in different countries across the globe and provided remote support
	\end{rSubsection}
	
	\begin{rSubsection}{ITBrat}{Jul 2015 -- Feb 2016}{Algorithmic trading startup, Moscow, Russia}{}
		\item Developed algorithmic trading application from initial discussion with the team to deployment and supporting
		\item Added low-level user-space networking to the project which allowed to decrease latency and increase profit
		\item Responsible for the performance of the code
%		\item Designed and supported the environment for the algorithm: build stage, version control, performance analysis using network dumps
	\end{rSubsection}
	
	%\begin{rSubsection}{\href{http://phobia.ru}{Claustrophobia}}{July 2014 -- Feb 2015}{Developer}{Moscow, Russia}
	%	\item Created system architecture for the real-world escape room game
	%	\item Implemented the solution using C++ (Atmel AVR, Linux)
	%	\item Results description: \url{habr.ru/p/258585/} (in Russian)
	%\end{rSubsection}
	
\end{rSection}

\begin{rSection}{Projects}
	\begin{rSubsection}{Quadcopter drone from scratch project}{2012 -- 2014}{}{}
		\item Developed \faExternalLink~\href{https://github.com/it-workshop/Quadrocopter}{an algorithm} in C++ for stabilization of a quadcopter drone from scratch using AVR microcontrollers, IMU sensors and PID regulators
		\item Managed the project consisting of 2-5 developers
		\item Conducted the analysis of launches to improve flying quality
		\item Results were published as a \faExternalLink~\href{http://web.archive.org/web/20141016114551/http://habrahabr.ru/company/technoworks/blog/216437/}{popular science article} {\em (in Russian)}
	\end{rSubsection}
\end{rSection}

\begin{rSection}{Scholarships}
	\vspace{-1em}
	\item \faExternalLink~\href{https://ic.epfl.ch/ResearchScholars}{Research Scholars}, a paid Research Assistant position, Swiss Federal Institute of Technology in Lausanne (EPFL), 2017 -- 2019
	\item \faExternalLink~\href{http://phystech-foundation.org/}{Abramov Fund's} scholarship for excellent grades, 2014
\end{rSection}

\begin{rSection}{Conferences}
\vspace{-1em}
\item \faExternalLink~\href{https://project.inria.fr/paiss/}{P.A.I.S.S.} (AI Summer School) (INRIA Grenoble, 2018){, \em participant of the practical sections given by top experts; \faExternalLink~\href{http://www.europe.naverlabs.com/Blog/Students-at-PAISS}{selected} to receive financial help }
\item \faExternalLink~\href{http://deepbayes.ru}{DeepBayes} school on Bayesian methods in Deep Learning (Moscow, 2017){, \em participant of lectures and practical sessions on Bayesian Methods}
\item \faExternalLink~\href{http://iitp.ru/en/conferences/itas}{Information Technologies and Systems} (Saint-Petersburg, Repino, 2016){, \em speaker, poster presenter}
%\item \faExternalLink~\href{https://sites.google.com/site/traditionalschool/about}{School} ``Control, Information, Optimization'' (Saint-Petersburg, Repino, 2016){, \em Poster presenter}
\end{rSection}

%\begin{rSection}{Courses}
%	\item Getting Started with Deep Learning (NVIDIA Deep Learning Workshop, MIPT, Feb 2017)
%	\item Approaches to Object Detection using DIGITS (NVIDIA Deep Learning Workshop, MIPT, Feb 2017)
%\end{rSection}

\begin{rSection}{Competitions}
\vspace{-1em}
\item \faExternalLink~\href{http://web.archive.org/web/20170224094223/http://rl.deephack.me/}{DeepHack.RL} hackathon on Deep Reinforcement Learning for Atari games, managed the team and developed an \faExternalLink~\href{https://github.com/sergeivolodin/deephack.rl}{evolutionary algorithm with an autoencoder} to solve Atari games, MIPT, Moscow, Russia, 2017
\end{rSection}

\begin{rSection}{Interests}
	Effective Altruism, Philosophy, Running ($1/2$ marathon 2018), Snowboarding, Swimming
\end{rSection}

\begin{rSection}{Volunteering}
	\begin{rSubsection}{Anti-corruption foundation}{2017}{A non-profit aimed at investigating corruption, Moscow, Russia}{}
Conveyed the results of the investigations by talking to people on the streets as a volunteer
	\end{rSubsection}
\end{rSection}

% failed applications for keeping the Balance of the Force
% OpenAI (2017, 2018)
% Anti-corruption foundation hackathon 2017
% Google Software Dev (2016, 2017, 2018)
% Google Research (2018)
% Facebook Software Dev 2017
% Facebook dinner 2018
% Microsoft Research 2017
% Samsung Research 2017
% Hyperloop EPFL Software Dev 2018
% LauzHack 2018
% Effective Altruism Global 2017
% 80000 career counceling 2017
% Research Scholars MLO 2017
% Summer@EPFL 2017
% Summer OIST 2017
% Summer ETHZ 2017
% Internship ELCA 2017
% Failed GRE Subject math, GRE
% Missed TOEFL

\end{document}
