\documentclass{resume} % Use the custom resume.cls style
\usepackage{fontawesome}
\usepackage[hidelinks]{hyperref}
\usepackage[left=0.75in,top=0.5in,right=0.75in,bottom=0.3in]{geometry} % Document margins

\name{Sergei Volodin}
\address{sergei.volodin@epfl.ch}
\address{+41 78 732 01 34}
\address{Rue du Verneret 10A, 1373 Chavornay, Vaud, Switzerland}
%\address{Birth date: 3 October 1994}

\begin{document}
\begin{rSection}{Links}
\item \faExternalLink~\href{http://linkedin.com/in/sergeivolodin/}{Linkedin}, \faExternalLink~\href{github.com/sergeivolodin}{Github}: sergeivolodin
\item \faExternalLink~\href{https://www.facebook.com/sergeivolodinepfl}{Facebook}: sergeivolodinepfl
\item Skype: sergeyvolodinmipt
\end{rSection}

\begin{rSection}{Education}
\begin{rSubsection}{\bf \'Ecole Polytechnique F\'ed\'erale de Lausanne}{Sep 2017 -- Jun 2019}{MSc in Computer Science}{Lausanne, Switzerland}
\item School of Computer and Communication Sciences
\item Relevant courses: Set theory, Machine Learning, Functional Programming (Scala), Software Engineering (Android, Scrum).
\end{rSubsection}

\begin{rSubsection}{\bf Moscow Institute of Physics and Technology}{Sep 2012 -- Jun 2017}{BSc in Applied Mathematics}{Moscow, Russia}
\item Department of Control and Applied Mathematics
\item \faExternalLink~\href{http://www.machinelearning.ru/}{Major} in {\bf Machine Learning}
\item Relevant courses: Algorithms and Data Structures, Functional analysis, Random processes, Convex Optimization.
\item GPA: {\bf 4.84}/5.00
\end{rSubsection}
\end{rSection}

\begin{rSection}{Research interests}
\begin{enumerate}
\item Artificial Intelligence; Machine Learning; Reinforcement Learning
\item Mathematical Optimization
\end{enumerate}
\end{rSection}

\begin{rSection}{Research experience}
	\begin{rSubsection}{EPFL, CHILI lab}{Sep 2017 -- present}{Research Assistant}{Lausanne, Switzerland}
		\item Created a website collecting a dataset for French BHK test to help dysgraphic children
		\item Researched into ways of adding Augmented Reality to the Cellulo project
	\end{rSubsection}
	
	\begin{rSubsection}{Skoltech, Center for Energy Systems}{Sep 2016 -- Jul 2017}{Research Intern}{Moscow, Russia}
%		\item Worked on the Power Flow Feasibility problem with Prof. Anatoly Dymarsky, Dr. Elena Gryazina, and Prof. Boris T. Polyak
		\item Designed and implemented the algorithm for cutting convex parts of the image of a quadratic map
		\item Examined the structure of the set of nonconvexities in Matlab
	\end{rSubsection}
	
	\begin{rSubsection}{MIPT, chair of Data Analysis}{Feb 2016 -- Jul 2016}{Undergraduate student}{Moscow, Russia}
		\item Compared machine learning algorithms for the ligand-receptor interaction problem
		\item Implemented Probabilistic Classifier Chains algorithm using scikit-learn library
%		\item The results were published in ITAS proceedings
	\end{rSubsection}
	
	\begin{rSubsection}{MIPT, chair of Theoretical Mechanics}{Oct 2012 -- Feb 2013}{Technician}{Moscow, Russia}
%		\item Worked on the paper ``Janibekov's effect and the laws of mechanics'' with A.G. Petrov
		\item Designed and implemented numerical simulations for Euler's rotation equations
		\item Checked soundness of the approximation using symbolic computations in Wolfram Mathematica
%		\item The results were published in Doklady Akademii Nauk
	\end{rSubsection}
\end{rSection}

\begin{rSection}{Publications}
%\item On the feasibility for the system of quadratic equations, Anatoly Dymarsky, Elena Gryazina, Boris Polyak, {\bf Sergei Volodin} {\em (expected)}
\item {\bf Volodin S.}, Popova M., Strijov V. Probabilistic prediction of nuclear receptors’ biological activity.\\ Proceedings of ITaS, 2016. \faExternalLink~\href{http://itas2016.iitp.ru/pdf/1570303389.pdf}{PDF}
\item Petrov A., {\bf Volodin S.} Janibekov’s effect and the laws of mechanics.\\ Doklady Akademii Nauk, 2013. \faExternalLink~\href{https://link.springer.com/article/10.1134/S1028335813080041}{PDF}
\end{rSection}

\begin{rSection}{Conferences}
\item \faExternalLink~\href{http://iitp.ru/en/conferences/itas}{Information Technologies and Systems} (Saint-Petersburg, Repino, 2016){, \em Speaker}
\item \faExternalLink~\href{https://sites.google.com/site/traditionalschool/about}{School} ``Control, Information, Optimization'' (Saint-Petersburg, Repino, 2016){, \em Poster presenter}
\item \faExternalLink~\href{http://deepbayes.ru}{DeepBayes} school on Bayesian methods in Deep Learning (Moscow, 2017){, \em Participant}
\end{rSection}

\begin{rSection}{Skills}
	\item {\bf Scientific programming:} numpy, scikit-learn, MATLAB, Mathematica, TensorFlow, Theano, R
	\item {\bf Programming:} C/C++, Python, AVR C++, Qt, Scala, Java, nasm, MS SQL
	\item {\bf Frameworks:} Qt, Django, Android Studio
	\item {\bf Environment:} Git, Bash, Debian Linux, Ubuntu, SVN
	\item {\bf Languages:} Russian (native), English (TOEFL iBT 112/120), French (beginner)
\end{rSection}

\begin{rSection}{Scholarships}
\item \faExternalLink~\href{http://phystech-foundation.org/}{Abramov Fund's scholarship} for excellent grades (2014)
\item \faExternalLink~\href{https://ic.epfl.ch/ResearchScholars}{Research Scholars} program at EPFL CHILI Lab (2017)
\end{rSection}

%\begin{rSection}{Courses}
%	\item Getting Started with Deep Learning (NVIDIA Deep Learning Workshop, MIPT, Feb 2017)
%	\item Approaches to Object Detection using DIGITS (NVIDIA Deep Learning Workshop, MIPT, Feb 2017)
%\end{rSection}

\begin{rSection}{Olympiads and hackathons}
%\item Google HashCode 2017 (participation)
\item \faExternalLink~\href{http://rl.deephack.me/}{DeepHack.RL} hackathon (Deep RL for Atari games), MIPT, Moscow, Russia, 2017. 4th place
%\item Sixteen interuniversity programming olympiad, Vologda, 2013\\
%\hfill {\em \url{http://olympiads.vologda-uni.ru/interuni/}}
\item \faExternalLink~\href{https://vk.com/devcup}{DevCup} software development competition, Moscow, Russia, 2013. 2nd place
\end{rSection}


\begin{rSection}{Work experience}
%Gap year Jan 2015 -- Feb 2016
\begin{rSubsection}{ITBrat}{Jul 2015 -- Feb 2016}{Software Engineer}{Moscow, Russia}
	\item Developed High Frequency Trading (cross-border arbitrage) application in C++, from initial discussion with the team to deployment and supporting
	\item Added low-level networking to the project using Solarflare OpenOnload library and hardware
	\item Designed and supported the environment for the algorithm: build stage, version control,\\ performance analysis using network dumps
\end{rSubsection}
	
\begin{rSubsection}{\faExternalLink~\href{http://escapecontrol.ru}{EscapeControl}}{Jul 2015 -- Feb 2016}{Software Engineer}{Moscow, Russia}
	\item Created \faExternalLink~\href{http://habr.ru/p/258585/}{system architecture} for the real-world escape room games
	\item Implemented the solution using C++ (Atmel AVR, Linux)
	\item Created a startup selling software \& hardware framework for real-world escape games
	\item Managed a team of two web developers
	\item Ten solutions sold, currently running in different countries
\end{rSubsection}
%\begin{rSubsection}{\href{http://phobia.ru}{Claustrophobia}}{July 2014 -- Feb 2015}{Developer}{Moscow, Russia}
%	\item Created system architecture for the real-world escape room game
%	\item Implemented the solution using C++ (Atmel AVR, Linux)
%	\item Results description: \url{habr.ru/p/258585/} (in Russian)
%\end{rSubsection}
		
\end{rSection}

\begin{rSection}{Projects}
\begin{rSubsection}{Quadcopter stabilization}{}{}{}
\item Developed an algorithm in C++ for stabilization of a quadcopter drones
\item Conducted the analysis of launches to improve flying quality
\item Results were \faExternalLink~\href{http://habr.ru/p/216437}{published} in the Habrahabr CS blog
\end{rSubsection}
\end{rSection}

\begin{rSection}{Volunteering}
	\begin{rSubsection}{\faExternalLink~\href{https://fbk.info/english/about/}{Anti-corruption foundation} (Alexey Navalny)}{}{}{}
		\item Donator (2015--2017)
		\item Rally participant (June 2017)
		\item Agitation volunteer (July 2017)
	\end{rSubsection}
\end{rSection}

%\begin{rSection}{Hobby}
%Swimming
%\end{rSection}

\end{document}
