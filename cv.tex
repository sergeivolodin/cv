%%%%%%%%%%%%%%%%%%%%%%%%%%%%%%%%%%%%%%%%%
% Medium Length Professional CV
% LaTeX Template
% Version 2.0 (8/5/13)
%
% This template has been downloaded from:
% http://www.LaTeXTemplates.com
%
% Original author:
% Trey Hunner (http://www.treyhunner.com/)
%
% Important note:
% This template requires the resume.cls file to be in the same directory as the
% .tex file. The resume.cls file provides the resume style used for structuring the
% document.
%
%%%%%%%%%%%%%%%%%%%%%%%%%%%%%%%%%%%%%%%%%

%----------------------------------------------------------------------------------------
%	PACKAGES AND OTHER DOCUMENT CONFIGURATIONS
%----------------------------------------------------------------------------------------

\documentclass{resume} % Use the custom resume.cls style
\usepackage{hyperref}
\usepackage[left=0.75in,top=0.5in,right=0.75in,bottom=0.3in]{geometry} % Document margins

\name{Sergei Volodin} % Your name
\address{sergei.volodin@phystech.edu, +7 916 600-90-58} % Your phone number and email
\address{Russia, Moscow} % Your address
\address{Birth date: 3 October 1994} % Your address

\begin{document}

%----------------------------------------------------------------------------------------
%	EDUCATION SECTION
%----------------------------------------------------------------------------------------

\begin{rSection}{Education}

{\bf Moscow Institute of Physics and Technology}, BSc student \hfill {\em Sep 2012 -- June 2017} \\ 
Department of Control and Applied Mathematics,\\sub-department of \href{http://www.machinelearning.ru/}{Intellectual Systems and Data Analysis}\\
GPA for 7 semesters: 8.98/10

\end{rSection}

%----------------------------------------------------------------------------------------
%	WORK EXPERIENCE SECTION
%----------------------------------------------------------------------------------------

\begin{rSection}{Research interests}
\begin{enumerate}
\item Artificial Intelligence
\item Machine Learning
\item Optimization
\end{enumerate}
\end{rSection}

\begin{rSection}{Research experience}
	\begin{rSubsection}{Skoltech, Center for Energy Systems}{Sep 2016 -- Present}{Research Intern}{Russia, Moscow}
		\item Worked on the Power Flow Feasibility problem with Assist. Prof. Anatoly Dymarsky and Dr. Elena Gryazina
		\item Designed (partially) and implemented the algorithm for cutting convex parts of the image in Matlab
		\item Examined the structure of the set of nonconvexities
		\item Results: \em article expected in 2017
	\end{rSubsection}
	
	\begin{rSubsection}{MIPT, sub-department of Data Analysis}{Feb 2016 -- July 2016}{Student}{Russia, Moscow}
		\item Worked on the ligand-receptor interaction problem using Machine Learning approach
		\item Implemented Probabilistic Classifier Chains algorithm using scikit-learn library
		\item Assessed this method as infeasible for the task
		\item Results: an article in ITAS proceedings
	\end{rSubsection}
	
	\begin{rSubsection}{MIPT, sub-department of Theoretical Mechanics}{Oct 2012 -- Feb 2013}{Technician}{Russia, Moscow}
		\item Worked on the article ``Janibekov's effect and the laws of mechanics'' with A.G. Petrov
		\item Designed and implemented numerical simulations for Euler's rotation equations
		\item Checked correctness of the approximation presented in the article using numerical simulation and symbolic computations in Wolfram Mathematica
		\item Results: an article in Doklady Akademii Nauk
	\end{rSubsection}
\end{rSection}

%----------------------------------------------------------------------------------------
%	EXAMPLE SECTION
%----------------------------------------------------------------------------------------

\begin{rSection}{Publications}
%\item On the feasibility for the system of quadratic equations, Anatoly Dymarsky, Elena Gryazina, Boris Polyak, {\bf Sergei Volodin} {\em (expected)}
\item Probabilistic prediction of nuclear receptors’ biological activity, {\bf Sergey Volodin}, Maria Popova, Vadim Strijov, ITAS 2016
\item Janibekov’s effect and the laws of mechanics, A.G. Petrov, {\bf S.E. Volodin}, 2013, published in Doklady Akademii Nauk, 2013, Vol. 451, No. 4, pp. 399–403.
\end{rSection}

\newpage

\begin{rSection}{Conferences}
\item Information Technologies and Systems (Saint-Petersburg, Repino, September 2016)\\
{\em Speaker}
\item Eights Traditional school “Control, Information, Optimization” (Saint-Petersburg, Repino, June 2016)\\
{\em Poster presenter}
\end{rSection}

\begin{rSection}{Scholarships}
\item \href{http://phystech-foundation.org/}{``Abramov fund scholarship''} for excellent grades (2014)
\end{rSection}

\begin{rSection}{Skills}
\item Programming: C/C++, Python, Matlab, scikit-learn, numpy, Mathematica, AVR C/C++,\\ x86 assembly (nasm), Microsoft SQL
\item Languages: Russian (native), English (B2)
\end{rSection}

\begin{rSection}{Courses}
	\item Getting Started with Deep Learning (NVIDIA Deep Learning Workshop, MIPT, Feb 2017)
	\item Approaches to Object Detection using DIGITS (NVIDIA Deep Learning Workshop, MIPT, Feb 2017)
	%\item ML Coursera
	%\item AT Coursera
	%\item PGM Coursera
\end{rSection}

\begin{rSection}{Olympiads and hackathons}
%\item Google HashCode 2017 (participation)
\item \href{http://rl.deephack.me/}{DeepHack.RL} hackathon (Deep RL for Atari games), MIPT, 2017. 4'th place (z-score)
%\item Sixteen interuniversity programming olympiad, Vologda, 2013\\
%\hfill {\em \url{http://olympiads.vologda-uni.ru/interuni/}}
\item DevCup, Russia, Moscow 2013, 2nd place (with BBC\&N team)\\
\hfill {\em \url{https://vk.com/devcup}}
\end{rSection}


\begin{rSection}{Professional experience}
%Gap year Jan 2015 -- Feb 2016
\begin{rSubsection}{ITBrat}{July 2015 -- Feb 2016}{Developer}{Russia, Moscow}
	\item Developed High Frequency Trading (cross-border arbitrage) application in C++, from initial discussion with the team to deployment and supporting
	\item Added low-level networking to the project using Solarflare OpenOnload library and hardware
	\item Designed and supported the environment for the algorithm: build stage, version control,\\ performance analysis using network dumps
\end{rSubsection}
	
\begin{rSubsection}{\href{http://escapecontrol.ru}{EscapeControl}}{July 2015 - Feb 2016}{Developer}{Russia, Moscow}
	\item Created a startup selling software \& hardware framework for real-world escape games
	\item Managed team of 2 web developers
	\item Ten copies sold, currently running in different countries
\end{rSubsection}
\newpage
\begin{rSubsection}{\href{http://phobia.ru}{Claustrophobia}}{July 2014 -- Feb 2015}{Developer}{Russia, Moscow}
	\item Created system architecture for the real-world escape room game
	\item Implemented the solution using C++ (Atmel AVR, Linux)
	\item Results description: \url{https://habrahabr.ru/company/technoworks/blog/258585/} (in Russian)
\end{rSubsection}
		
\end{rSection}
%\newpage

\begin{rSection}{Hobby}
\begin{rSubsection}{Quadcopter stabilization}{2012 -- 2014}{}{Russia, Moscow}
\item Worked with TechnoWorks team (\href{http://www.pm.inf.ethz.ch/people/person-detail.html?persid=220074}{Arshavir Ter-Gabrielyan} and others)
\item Developed the algorithm for stabilization a quadcopter drone using C++ (Atmel AVR, Linux)
\item Conducted the analysis of launches to improve flying quality
\item Results description: \url{https://habrahabr.ru/company/technoworks/blog/216437/} (in Russian)
\end{rSubsection}

\end{rSection}

%----------------------------------------------------------------------------------------

\end{document}
