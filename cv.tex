%%%%%%%%%%%%%%%%%%%%%%%%%%%%%%%%%%%%%%%%%
% Medium Length Professional CV
% LaTeX Template
% Version 2.0 (8/5/13)
%
% This template has been downloaded from:
% http://www.LaTeXTemplates.com
%
% Original author:
% Trey Hunner (http://www.treyhunner.com/)
%
% Important note:
% This template requires the resume.cls file to be in the same directory as the
% .tex file. The resume.cls file provides the resume style used for structuring the
% document.
%
%%%%%%%%%%%%%%%%%%%%%%%%%%%%%%%%%%%%%%%%%

%----------------------------------------------------------------------------------------
%	PACKAGES AND OTHER DOCUMENT CONFIGURATIONS
%----------------------------------------------------------------------------------------

\documentclass{resume} % Use the custom resume.cls style
\usepackage{hyperref}
\usepackage[left=0.75in,top=0.5in,right=0.75in,bottom=0.3in]{geometry} % Document margins

\name{Sergei Volodin} % Your name
\address{sergei.volodin@phystech.edu} % Your phone number and email
\address{+7 916 600-90-58}
\address{Russia, Moscow} % Your address

\begin{document}

%----------------------------------------------------------------------------------------
%	EDUCATION SECTION
%----------------------------------------------------------------------------------------

\begin{rSection}{Education}

{\bf Moscow Institute of Physics and Technology} \hfill {\em June 2017} \\ 
Department of Control and Applied Mathematics\\
GPA for 6 semesters: 9.2/10

\end{rSection}

%----------------------------------------------------------------------------------------
%	WORK EXPERIENCE SECTION
%----------------------------------------------------------------------------------------

\begin{rSection}{Research interests}
%\begin{enumerate}
	\item Artificial intelligence
	\item Machine learning
	\item Optimization
%\end{enumerate}

\end{rSection}

%----------------------------------------------------------------------------------------
%	EXAMPLE SECTION
%----------------------------------------------------------------------------------------

\begin{rSection}{Publications}
	\item Janibekov’s effect and the laws of mechanics, A.G. Petrov, S.E. Volodin, 2013, published in Doklady Akademii Nauk, 2013, Vol. 451, No. 4, pp. 399–403.
	
	\item Probabilistic prediction of nuclear receptors’ biological activity, Sergey Volodin, Maria Popova, Vadim Strijov, ITAS 2016
	
\end{rSection}

\begin{rSection}{Conferences}
\item Eights Traditional school “Control, Information, Optimization”\\
{\em Poster presenter}
\item Information Technologies and Systems 2016\\
{\em Speaker}
\end{rSection}

\begin{rSection}{Skills}
Programming: C++, C, Python, AVR C/C++, x86 assembly, Microsoft SQL\\
Languages: Russian (native), English
\end{rSection}


\begin{rSection}{Achievements}
\item Sixteen interuniversity programming olympiad, Vologda, 2013, Successful performance \\
\hfill {\em \url{http://olympiads.vologda-uni.ru/interuni/}}
\item DevCup, Russia, Moscow 2013, 2nd place (with BBC\&N team)\\
\hfill {\em \url{https://vk.com/devcup}}
\end{rSection}
\newpage

\begin{rSection}{Experience}
	
	\begin{rSubsection}{Claustrophobia}{July 2014 -- Feb 2015}{Developer}{Russia, Moscow}
		\url{https://habrahabr.ru/company/technoworks/blog/258585/}
		\item C++ low-level programming (Atmel AVR)
		\item C++ programming (Linux)
		\item System architecture
	\end{rSubsection}
	
	%------------------------------------------------
	
	\begin{rSubsection}{ITBrat}{July 2015 -- Feb 2016}{Developer}{Russia, Moscow}
		\item High Frequency Trading application development
		\item Low-level networking in Linux
	\end{rSubsection}
	\begin{rSubsection}{Skoltech}{Sep 2016 --}{Research Intern}{Russia, Moscow}
		\item Power flow feasibility problem
	\end{rSubsection}
	
%	\begin{rSubsection}{EscapeControl}{July 2015 - Feb 2016}{CEO}{Russia, Moscow}
%		\item C++ low-level programming
%		\item Linux administration
%		\item System architecture
%	\end{rSubsection}
	
\end{rSection}

\begin{rSection}{Hobby}
	Quadcopter stabilization development 2012--2014\\
	\url{https://habrahabr.ru/company/technoworks/blog/216437/}
\end{rSection}

%----------------------------------------------------------------------------------------

\end{document}
