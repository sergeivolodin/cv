%%%%%%%%%%%%%%%%%%%%%%%%%%%%%%%%%%%%%%%%%
% Medium Length Professional CV
% LaTeX Template
% Version 2.0 (8/5/13)
%
% This template has been downloaded from:
% http://www.LaTeXTemplates.com
%
% Original author:
% Trey Hunner (http://www.treyhunner.com/)
%
% Important note:
% This template requires the resume.cls file to be in the same directory as the
% .tex file. The resume.cls file provides the resume style used for structuring the
% document.
%
%%%%%%%%%%%%%%%%%%%%%%%%%%%%%%%%%%%%%%%%%

%----------------------------------------------------------------------------------------
%	PACKAGES AND OTHER DOCUMENT CONFIGURATIONS
%----------------------------------------------------------------------------------------

\documentclass{resume} % Use the custom resume.cls style

\usepackage[russian]{babel}
\usepackage[utf8]{inputenc}
\usepackage[left=0.75in,top=0.3in,right=0.75in,bottom=0.3in]{geometry} % Document margins

\name{Сергей Володин} % Your name
\address{sergei.volodin@phystech.edu} % Your phone number and email
\address{+7 916 600-90-58}
\address{Россия, Москва} % Your address

\begin{document}

%----------------------------------------------------------------------------------------
%	EDUCATION SECTION
%----------------------------------------------------------------------------------------

\begin{rSection}{Образование}

{\bf Московский физико-технический институт} \hfill {\em Июнь 2017} \\ 
Факультет управления и прикладной математики\\
Средний балл за 6 семестров: 9.2/10

\end{rSection}

%----------------------------------------------------------------------------------------
%	WORK EXPERIENCE SECTION
%----------------------------------------------------------------------------------------

\begin{rSection}{Исследовательские интересы}
%\begin{enumerate}
	\item Искуственный интеллект
	\item Машинное обучение
	\item Оптимизация
%\end{enumerate}

\end{rSection}

%----------------------------------------------------------------------------------------
%	EXAMPLE SECTION
%----------------------------------------------------------------------------------------

\begin{rSection}{Публикации}
	\item Janibekov’s effect and the laws of mechanics, A.G. Petrov, S.E. Volodin, 2013, published in Doklady Akademii Nauk, 2013, Vol. 451, No. 4, pp. 399–403.
	
	\item Probabilistic prediction of nuclear receptors’ biological activity, Sergey Volodin, Maria Popova, Vadim Strijov, Не опубликована
	
\end{rSection}

\begin{rSection}{Конференции}
\item Восьмая традиционная школа <<Управление, информация и оптимизация>>\\
{\em Постерная сессия}
\end{rSection}

\begin{rSection}{Навыки}
Программирование: C++, C, Python, AVR C/C++, ассемблер x86, Microsoft SQL\\
Языки: Русский, Английский
\end{rSection}


\begin{rSection}{Достижения}
\item Шестнадцатая межвузовская олимпиада по программированию, Вологда, 2013, Успешное выступление \\
\hfill {\em http://olympiads.vologda-uni.ru/interuni/}
\item DevCup, Москва 2013, II место (в команде BBC\&N)\\
\hfill {\em https://vk.com/devcup}
\end{rSection}


\begin{rSection}{Опыт}
	
%	\begin{rSubsection}{Klaustrophobia}{July 2014 - Feb 2015}{Developer}{Russia, Moscow}
%		\item C++ low-level programming
%		\item System architecture
%	\end{rSubsection}
	
	%------------------------------------------------
	
	\begin{rSubsection}{ITBrat}{July 2015 - Feb 2016}{Разработчик}{Москва}
		\item Разработка приложения в сфере High Frequency Trading
		\item Низкоуровневая работа с сетью в Linux
	\end{rSubsection}
	
%	\begin{rSubsection}{EscapeControl}{July 2015 - Feb 2016}{CEO}{Russia, Moscow}
%		\item C++ low-level programming
%		\item Linux administration
%		\item System architecture
%	\end{rSubsection}
	
\end{rSection}

\begin{rSection}{Хобби}
	Разработка ПО для управления квадрокоптером 2012--2014\\
	https://habrahabr.ru/company/technoworks/blog/216437/
\end{rSection}

%----------------------------------------------------------------------------------------

\end{document}
